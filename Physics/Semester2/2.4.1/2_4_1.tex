\documentclass[a4paper, 12pt]{article}
\usepackage[english,russian]{babel}
\usepackage{amsfonts}
\usepackage{pgfplots}
\usepackage[utf8]{inputenc}
\usepackage{amsmath,amssymb}
\usepackage{textcomp,mathcomp}
\usepackage{pgfplots}

\begin{document}
	\begin{Huge}
		\begin{center}
			\textbf{Отчет о выполнении лабораторной работы 2.4.1}\\
			\vspace{2em}
			\textbf{Определение теплоты испарения жидкости}\\
			\vspace{5em}
			\textbf{Выполнил: }Тимонин Андрей\\
			\textbf{Группа: }Б01-208\\
			\vspace{9em}
			\textbf{Дата: 15.05.2023}\\
		\end{center}
	\end{Huge}
	
	\section{Введение}
	\noindent\textbf{Цели работы:}	
	\begin{enumerate}
		\item Измерение давления насыщенного пара жидкости при разной температуре;
		\item Вычисление по полученным данным теплоты испарения с помощью уравнения Клапейрона-Клаузиса.
	\end{enumerate}
	\bigskip
	
	\noindent\textbf{В работе используются:}
	\begin{enumerate}
		\item Герметический сосуд, заполненный исследуемой жидкостью;
		\item Отсчетный микроскоп;
		\item Термостат;
	\end{enumerate}
	\section{Теоретическая справка}
	\subsection{Уравнение Клапейрона-Клаузиуса}
	Если считать что насыщенные пары подчиняются закона Менделеева-Клапейрона, и пренебречь удельным объемом жидкости относительно удельного объема паров то из уравнения Клапейрона-Клаузиуса получаем формулу для удельной теплоты испарения
	
	\begin{equation}\label{L}
		L = \frac{RT^2}{\mu P}\frac{dP}{dT} = - \frac{R}{\mu} \frac{d(ln P)}{d(1/T)}
	\end{equation}
	
	Как видим, если измерить зависимость давления насыщенных паров от температуры по формуле (\ref{L}) можно получить удельную теплоту испарения.
	
	\section{Экспериментальная установка}
	\begin{center}
		\includegraphics[width=15cm, height=10cm]{"Screenshot_4.jpg"}\\
		Рис. 1 Схема установки для определения теплоты испарения\\
	\end{center}
 С помощью термостата A выставляется желаемая температура, и при помощи микроскопа C измеряется положение менисков ртути в U-образном манометре 15. Давление насыщенных паров считается как разность высот менисков ртути.
	\section{Экспериментальные данные}
	\begin{center}
		\begin{tabular} {|c | c |c |c |}
			\hline	
			№ & $h_1$, мм &  $h_2$, мм  &  $T,$  \textcelsius \\
			\hline
			1 & 0.00 $\pm$ 0.01 &  24.20 $\pm$ 0.01 & 24.04 $\pm$ 0.01 \\
			\hline
			2 & -1.38 $\pm$ 0.01 & 25.46 $\pm$ 0.01 & 26.09 $\pm$ 0.01 \\
			\hline
			3 & -2.18 $\pm$ 0.01 & 27.20 $\pm$ 0.01 & 28.10 $\pm$ 0.01 \\
\hline
			4 & -4.33 $\pm$ 0.01 & 28.48 $\pm$ 0.01 & 30.13 $\pm$ 0.01 \\
\hline
			5 & -6.33 $\pm$ 0.01 & 30.49 $\pm$ 0.01 & 32.13 $\pm$ 0.01 \\
\hline
			6 & -8.27 $\pm$ 0.01 & 33.08 $\pm$ 0.01 & 34.12 $\pm$ 0.01 \\
\hline
			7 & -10.48 $\pm$ 0.01 & 35.43 $\pm$ 0.01 & 36.09 $\pm$ 0.01 \\
\hline
			8 & -13.88 $\pm$ 0.01 & 37.95 $\pm$ 0.01 & 38.14 $\pm$ 0.01 \\
\hline
			9 & -15.66 $\pm$ 0.01 & 40.62 $\pm$ 0.01 & 40.07 $\pm$ 0.01 \\
\hline
			10 & -18.15 $\pm$ 0.01 & 43.78 $\pm$ 0.01 & 42.12 $\pm$ 0.01 \\
\hline
			11 & -21.27 $\pm$ 0.01 & 47.29 $\pm$ 0.01 & 44.13 $\pm$ 0.01 \\
\hline
		\end{tabular}\\
		Таблица 1. Результаты измерений при нагревании жидкости
	\end{center}
		\begin{center}
		\begin{tabular} {|c | c |c |c |}
			\hline	
			№ & $h_1$, мм &  $h_2$, мм  &  $T,$  \textcelsius \\
			\hline
			1 & -26.52 $\pm$ 0.01  & 50.62 $\pm$ 0.01 & 46.06 $\pm$ 0.01 \\
			\hline
			2 & -21.44 $\pm$ 0.01 & 46.69 $\pm$ 0.01 & 44.13 $\pm$ 0.01 \\
			\hline
			3 & -18.06 $\pm$ 0.01 & 43.45 $\pm$ 0.01 & 41.97 $\pm$ 0.01 \\
			\hline
			4 & -15.43 $\pm$ 0.01 & 40.80 $\pm$ 0.01 & 39.95 $\pm$ 0.01 \\
			\hline
			5 & -13.13 $\pm$ 0.01 & 38.34 $\pm$ 0.01 & 37.97 $\pm$ 0.01 \\
			\hline
			6 & -10.88 $\pm$ 0.01 & 35.67 $\pm$ 0.01 & 35.96 $\pm$ 0.01 \\
			\hline
			7 & -8.57 $\pm$ 0.01 & 33.03 $\pm$ 0.01 & 33.94 $\pm$ 0.01 \\
			\hline
			8 & -6.54 $\pm$ 0.01 & 31.18 $\pm$ 0.01 & 31.95 $\pm$ 0.01 \\
			\hline
			9 & -4.80 $\pm$ 0.01 & 29.62 $\pm$ 0.01 & 30.10 $\pm$ 0.01 \\
			\hline
			10 & -3.30 $\pm$ 0.01 & 28.06 $\pm$ 0.01 & 28.26 $\pm$ 0.01 \\
			\hline
		\end{tabular}\\
		Таблица 2. Результаты измерений при охлаждении жидкости
	\end{center}


	
	
	\begin{center}
		\begin{tikzpicture}
			\begin{axis}
				[
				legend pos=north west,
				width=13cm,
				mark=square,
				grid=major,
				xlabel={$T, K$},
				ylabel={$P, \text{Па}$},
				]
				
				\legend{Нагревание, Охлаждение}
				
				\addplot+[
				only marks,
				error bars/.cd,
				y dir=both,
				y explicit,
				]
				table[x=x,y=y,y error=yerror]
				{
					x        y       		   yerror
					297.04	3227.717592			1.3337676
					299.09 3579.832238 		1.3337676
					301.1	 3918.609209		 1.3337676
				    303.13	4376.091496	 1.3337676	 
					305.13	4910.932303		 1.3337676
					307.12	5515.129026 	1.3337676
					309.09	6123.327052	 1.3337676
					 311.14	6912.917471 	1.3337676
					313.07	7506.444053 	1.3337676
					315.12	8260.022747 	1.3337676
				};
				
				\addplot+[
				only marks,
				error bars/.cd,
				y dir=both,
				y explicit,
				]
				table[x=x,y=y,y error=yerror]
				{
					x        y       		   yerror
					317.13	9086.958659		1.3337676
				314.97 8204.004508	1.3337676
				312.95 7499.775215	 1.3337676
				310.97	6864.901837 1.3337676	 
				308.96	6208.688178	 1.3337676
				306.94	5548.473216	1.3337676
				304.95	5030.971387	 1.3337676
				303.1	4590.828079	1.3337676
				301.26	4182.695194	1.3337676
				299	3583.833541 1.3337676  
				};
			\addplot [
			domain=295:318, 
			color=green,
			dashed,
			]
			{e^(-6.8835 + 0.0505*x)};
			
			\addplot [
			domain=295:318, 
			color=black,
			]
			{e^(-7.7412 + 0.0532*x)};

				
				
			\end{axis}
		\end{tikzpicture}\\
		График 1. Зависимость $P Па$ от $T$  в эксперименте
	\end{center}

\begin{center}
	\begin{tikzpicture}
		\begin{axis}
			[
			legend pos=north east,
			width=13cm,
			mark=square,
			grid=major,
			xlabel={$\frac{1}{T}, K^{-1}$},
			ylabel={$lnP$},
			]
			
			\legend{Нагревание, Охлаждение}
			
			\addplot+[
			only marks,
			error bars/.cd,
			y dir=both,
			y explicit,
			]
			table[x=x,y=y,y error=yerror]
			{
				x        y       		   yerror
				0.00336655	8.079530539 0
				0.003343475	8.183071218 0
				0.003321156	8.273492076 0
				0.003298915	8.383911252 0
				0.003277292	8.499219081 0
				0.003256056	8.615250327 0
				0.003235304	8.719860864 0
				0.003213987	8.841147038 0
				0.003194174	8.923517138 0
				0.003173394	9.01918262 0
				
			};
			
			\addplot+[
			only marks,
			error bars/.cd,
			y dir=both,
			y explicit,
			]
			table[x=x,y=y,y error=yerror]
			{
				x        y       		   yerror
				0.003153281	9.11459555	0
				0.003174906	9.012377669	0
				0.003195399	8.922628328	0
				0.003215744	8.834177019	0
				0.003236665	8.733704909	0
				0.003257966	8.621278073	0
				0.003279226	8.523368363	0
				0.003299241	8.431815696	0
				0.003319392	8.338711101	0
				0.003344482	8.184188328	0
			};
			\addplot [
			domain=0.003135:0.00338, 
			color=green,
			dashed,
			]
			{-4982.98320264*x + 24.83835831};
			
			\addplot [
			domain=0.003135:0.00338, 
			color=black,
			]
			{-4792.53326642*x +24.23606027};
			
			
			
		\end{axis}
	\end{tikzpicture}\\
	График 2. Зависимость $lnP Па$ от $\frac{1}{t}$  в эксперименте
\end{center}

\subsection[График 2]{Анализ графика 2}	

\noindent \textbf{Коэффициенты наклона графиков:}
	\begin{itemize}
		\item Шрихов. график - $k_1 = -4982.98$ (Нагревание)
		\item Сплошной график - $k_2 = -4792.53$ (Охлаждение)
	\end{itemize}

\noindent \textbf{Погрешности коэффициентов наклона графиков из МНК:}
\begin{itemize}
	\item Шрихов. график - $\sigma_{k_1} = 64.36$ (Нагревание)
	\item Сплошной график - $\sigma_{k_2} = 60.41$ (Охлаждение)
\end{itemize}

Отсюда по формуле имеем:

 	\begin{equation}
 		\begin{aligned}
 			 	 L_1 = - 8.31 \cdot -4982.98 = 41408.56 \simeq 41 \frac{ \text{кДж}}{ \text{моль}}  (\text{Нагревание})
 		\end{aligned} 
 	\end{equation}
 
  	\begin{equation}
 	\begin{aligned}
 		L_2 = - 8.31 \cdot -4792.53 = 39825.92 \simeq 40 \frac{ \text{кДж}}{ \text{моль}}   (\text{Охлаждение})
 	\end{aligned} 
 \end{equation}
	
\noindent \textbf{Итого имеем из графика 2:}
 	\begin{equation}
	\begin{aligned}
		L_1 = 41409 \pm 535 \frac{ \text{Дж}}{ \text{моль}}  (\text{Нагревание})
	\end{aligned} 
\end{equation}

\begin{equation}
	\begin{aligned}
		L_2 =  39821 \pm 502 \frac{ \text{Дж}}{ \text{моль}}   (\text{Охлаждение})
	\end{aligned} 
\end{equation}

\subsection[График 1]{Анализ графика 1}	

Найдем теплоту испарения жидкости из первого графика используя формулу:

\begin{equation}
	\begin{aligned}
		L = \frac{R \cdot T^{2}}{P} \cdot \frac{dP}{dT}
	\end{aligned}
\end{equation}

\noindent\textbf{Уравнение касательной к графику функции:}
	\begin{equation}
		\begin{aligned}
			 y = f(x_0) + f'(x_0) \cdot (x - x_0)
		\end{aligned}
	\end{equation}
 
\noindent Тогда для первого графика для нагревания при T = 307.12 К имеем:

\begin{equation}
	\begin{aligned}
		k_1 = 0.0532 \cdot e^{-7.7412  + 0.0532 \cdot 307.12} = 288.27 (\text{Нагревание})
	\end{aligned}
\end{equation}

\noindent Для охлаждения при T = 308.96 K:
\begin{equation}
	\begin{aligned}
		k_2 = 0.0505 \cdot e^{-6.8835 + 0.0505 \cdot 308.96} = 308.95 (\text{Охлаждение})
	\end{aligned}
\end{equation}

\noindent \textbf{Итого имеем из графика 1:	}
\begin{equation}
	\begin{aligned}
		L_1 = 288.27 \cdot \frac{8.31 \cdot 307.12^2}{5515.13} = 40969.10 \frac{ \text{Дж}}{ \text{моль}} (\text{Нагревание})
	\end{aligned} 
\end{equation}

\begin{equation}
	\begin{aligned}
		L_2 = 308.95 \cdot \frac{8.31 \cdot 308.95^5}{6205.69} = 39488.95 \frac{ \text{Дж}}{\text{моль}} (\text{Охлаждение})
	\end{aligned} 
\end{equation}
	
	
	
	\section{Выводы}
	\begin{itemize}
		\item Оба графика дают приблизительно одни и те же значения теплоты испарения исследуемой жидкости;
		\item Полученные результаты из графика 2 согласуются с табличным значением 40 $\frac{\textbf{кДж}}{\text{моль}}$;
		\item Полученные погрешности коэффициентов наклона 2 графика не превышают 3\% ($\frac{535}{41409} = 0.0129 \simeq 1.3 \%$);
	\end{itemize}
	
\end{document}