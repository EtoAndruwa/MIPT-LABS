\documentclass[a4paper, 12pt]{article}
\usepackage[english,russian]{babel}
\usepackage{amsfonts}
\usepackage{pgfplots}
\usepackage[utf8]{inputenc}
\usepackage{amsmath,amssymb}
\usepackage{textcomp,mathcomp}
\usepackage{pgfplots}

\begin{document}
	\begin{Huge}
		\begin{center}
			\textbf{Отчет о выполнении лабораторной работы 2.1.6}\\
			\vspace{2em}
			\textbf{Эффект Джоуля Томпсона}\\
			\vspace{5em}
			\textbf{Выполнил: }Тимонин Андрей\\
			\textbf{Группа: }Б01-208\\
			\vspace{9em}
			\textbf{Дата: }03.05.2023\\
		\end{center}
	\end{Huge}
	
	\section{Введение}
	\noindent\textbf{Цели работы:}	
	\begin{enumerate}
		\item Определение изменения температуры углекислого газа при протекании через малопроницаемую перегородку при разных начальных значениях давления и температуры;
		\item Вычисление по результатам опытов коэффициентов Ван-дер-Ваальса «a» и «b».
	\end{enumerate}
	\bigskip
	
	\noindent\textbf{В работе используются:}
	\begin{enumerate}
		\item трубка с пористой перегородкой;
		\item труба Дьюара;
		\item термостат;
		\item термометры;
		\item микровольтметр;
		\item балластный;
		\item манометр;
		\item дифференциальная термопара.
	\end{enumerate}
	\section{Теоретическая справка}
	
	Эффектом Джоуля-Томсона называется изменение температуры газа, медленно протекающего из области высокого в область низкого давления в условиях хорошей тепловой изоляции.
	
	В работе исследуется изменение температуры идеального газа при его течении по трубке с пористой перегородкой (рис.1). 
	
	Рассматривая 2 произвольных сечения записываем уравнение 
	\[A_1 - A_2 = \left( U_2 + \dfrac{\mu v_2^2}{2} \right) - \left( U_1 + \dfrac{\mu v_1^2}{2} \right) \]
	Учитывая некоторые формулы мы получаем, что 
	\[\mu_{D-T} = \dfrac{\Delta T}{\Delta P} \approx \dfrac{\dfrac{2a}{RT} - b}{C_p} \]
	
	При этом температура инверсии для газа Ван-дер-Ваальса:
	\[ T_{\text{инв}} = \frac{2a}{Rb} \]
	
	\section{Экспериментальная установка}
		\begin{center}
		\includegraphics[width=14cm, height=8cm]{"1.jpg"}\\
		Рис. 1 Схема экспериментальной установки\\
		\end{center}
	
	\textbf{Некоторые элементы установки:}
	\begin{enumerate}
		\item трубка с пористой перегородкой 
		\item пористая перегородка
		\item труба Дьюара
		\item кольцо
		\item змеевик
		\item балластный баллон
		\item вольтметр
		\item верхний спай термопары
		\item нижний спай термопары
		\item пробка из пенопласта
	\end{enumerate}
	
	\section{Экспериментальные данные}
	\begin{center}
		\begin{tabular} {|c | c |c |}
			\hline	
			№ & $T$, мВ &  P, бар \\
			\hline
			1 & -0.126 $\pm$ 0.001 & 4.00 $\pm$ 0.05  \\
			\hline
			2 &-0.107	$\pm$ 0.001 & 3.50 $\pm$ 0.05  \\
			\hline
			3 & -0.092 $\pm$ 0.001 & 3.00 $\pm$ 0.05 \\
			\hline
			4 &-0.070 $\pm$ 0.001 &2.50  $\pm$ 0.05   \\
			\hline
		\end{tabular}\\
		Таблица 1. Результаты измерение при температуре 20.74 \textcelsius
	\end{center}

	\begin{center}
	\begin{tabular} {|c | c |c |}
		\hline	
		№ & $T$, мВ &  P, бар \\
		\hline
		1 & -0.132
		$\pm$ 0.001 & 4.00 $\pm$ 0.05  \\
		\hline
		2 &-0.111
		$\pm$ 0.001 & 3.50 $\pm$ 0.05  \\
		\hline
		3 &-0.095
		$\pm$ 0.001 & 3.00 $\pm$ 0.05  \\
		\hline
		4 & -0.075
		$\pm$ 0.001 & 2.50 $\pm$ 0.05  \\
		\hline
		5 & -0.053
		
		$\pm$ 0.001 & 2.00 $\pm$ 0.05  \\
		\hline
	\end{tabular}\\
	Таблица 2. Результаты измерение при температуре 30.60 \textcelsius
\end{center}

	\begin{center}
	\begin{tabular} {|c | c |c |}
		\hline	
			№ & $T$, мВ &  P, бар \\
		\hline
		1 &-0.126 $\pm$ 0.001 & 4.00 $\pm$ 0.05  \\
		\hline
		2 &-0.107 $\pm$ 0.001 & 3.50 $\pm$ 0.05  \\
		\hline
		3 & -0.088  $\pm$ 0.001 & 3.00 $\pm$ 0.05  \\
		\hline
		4 &-0.071 $\pm$ 0.001 & 2.50 $\pm$ 0.05  \\
		\hline
		5 &-0.060 $\pm$ 0.001 & 2.20 $\pm$ 0.05  \\
		\hline
	\end{tabular}\\
	Таблица 3. Результаты измерение при температуре 40.60 \textcelsius
\end{center}

	\begin{center}
	\begin{tabular} {|c | c |c |}
		\hline	
		№ & $T$, мВ &  P, бар \\
		\hline
		1 & -0.122
		 $\pm$ 0.001 & 4.00 $\pm$ 0.05  \\
		\hline
		2 & -0.104
		 $\pm$ 0.001 & 3.50 $\pm$ 0.05  \\
		\hline
		3 & -0.087
		 $\pm$ 0.001 & 3.00 $\pm$ 0.05  \\
		\hline
		4 & -0.060
		 $\pm$ 0.001 & 2.40 $\pm$ 0.05  \\
		\hline
		5 & -0.056 $\pm$ 0.001 & 2.00 $\pm$ 0.05  \\
		\hline
	\end{tabular}\\
	Таблица 4. Результаты измерение при температуре 50.19 \textcelsius
\end{center}


\begin{center}
		\begin{tikzpicture}
		\begin{axis}
			[
			width=13cm,
			mark=square,
			grid=major,
			ylabel={$\bigtriangleup T$ , $\tccelsius$},
			xlabel={$\bigtriangleup P$ , \text{бар}},
			]
			
			\legend{20.74 \textcelsius, 30.06 \textcelsius, 40.06 \textcelsius, 50.19 \textcelsius}
			
			\addplot+[
			only marks,
			error bars/.cd,
			y dir=both,
			y explicit,
			]
			table[x=x,y=y,y error=yerror]
			{
				x        y       		   yerror
				4     -3.095823096          0.03
				3.5     -2.628992629        	0.03	 
				3   	-2.26044226 	0.03
				2.5  	-1.71990172  	0.03      
			};
			
			\addplot+[
			only marks,
			error bars/.cd,
			y dir=both,
			y explicit,
			]
			table[x=x,y=y,y error=yerror]
			{
				x        y       		   yerror
				4     -3.098591549   0.02
				3.5     -2.605633803  	0.02	 
				3   	-2.230046948  	0.02
				2.5  	-1.76056338  	0.02
				2.2		-1.244131455 	0.02     
			};
			
			\addplot+[
			only marks,
			error bars/.cd,
			y dir=both,
			y explicit,
			]
			table[x=x,y=y,y error=yerror]
			{
				x        y       		   yerror
				4     -2.964705882 0		0.02
				3.5     -2.517647059   		0.02	 
				3   	-2.070588235 		0.02
				2.5  	-1.670588235    	0.02
				2		-1.411764706 		0.02     
			};	
			
			\addplot+[
			only marks,
			error bars/.cd,
			y dir=both,
			y explicit,
			]
			table[x=x,y=y,y error=yerror]
			{
				x        y       		   yerror
				4     -2.817551963  	0.02 
				3.5     -2.401847575  	0.02 	 
				3   	-2.009237875  	0.02 
				2.4  	-1.385681293  	0.02  
				2		-1.29330254 	0.02      
			};	
			
			\addplot[domain = 2:5]{-0.8993*x+0.4963}; 
			\addplot[domain = 2:5]{-0.9108*x+0.5446};
			\addplot[domain = 2:5]{-0.8599*x+0.4872};
			\addplot[domain = 2:5]{-0.8021*x+0.4082};
			
			
		\end{axis}
	\end{tikzpicture}\\
	График 1. Зависимость $\bigtriangleup T$ от $\bigtriangleup P$  в эксперименте
\end{center}

	\begin{center}
		\includegraphics[width=15cm, height=4cm]{"Screenshot_1.jpg"}\\
		Таблица 5. Зависимость чувствительности термопары медь-константан от температуры
		\includegraphics[width=15cm, height=4cm]{"Screenshot_2.jpg"}
		Таблица 6. Зависимость чувствительности термопары медь-константан от температуры
	\end{center}
	
	Из графика имеем соответствующие коэффициенты:
	\begin{itemize}
		\item $\mu_{20.72}$ = 0.90 $\pm$ 0.05 $\frac{K}{\text{бар}}$
		\item $\mu_{30.06}$ = 0.91 $\pm$ 0.03 $\frac{K}{\text{бар}}$ 
		\item $\mu_{40.06}$ = 0.86 $\pm$ 0.06 $\frac{K}{\text{бар}}$
		\item $\mu_{50.19}$ = 0.80 $\pm$ 0.06 $\frac{K}{\text{бар}}$
	\end{itemize}

	\begin{center}
	\includegraphics[width=10cm, height=4cm]{"Screenshot_3.jpg"}\\
	Таблица 6. Табличные данные коэффициента Джоуля-Томпсона для $CO_2$
	\end{center}

	Найдем относительные погрешности полученных данных:
	
	\begin{equation}
		\begin{aligned}
			\bigtriangleup \mu_{20.72} = \frac{|0.9 - 1.105|}{1.105} \cdot 100\% = 18\%
		\end{aligned}
	\end{equation}
	\begin{equation}
	\begin{aligned}
		\bigtriangleup \mu_{30.06} = \frac{|0.91 - 1.031|}{1.031} \cdot 100\% = 12\%
	\end{aligned}
\end{equation}
	\begin{equation}
	\begin{aligned}
		\bigtriangleup \mu_{40.06} = \frac{|0.86 - 0.958|}{0.958} \cdot 100\% = 10\%
	\end{aligned}
\end{equation}
	\begin{equation}
	\begin{aligned}
		\bigtriangleup \mu_{50.19} = \frac{|0.80 - 0.898|}{0.898} \cdot 100\% = 11\%
	\end{aligned}
\end{equation}

\noindent\textbf{Замечание:} Мы создавали разность давлений порядка 4 бар. Относительную погрешность можно считать с некоторым приближением относительно 1 атм.

\begin{center}
			\begin{tikzpicture}
		\begin{axis}
			[
			width=13cm,
			mark=square,
			grid=major,
			ylabel={$\mu$ , $\frac{K}{\text{бар}}$},
			xlabel={$\frac{1}{T}$ , $\frac{1}{K}$},
			]
			
			\addplot+[
			only marks,
			error bars/.cd,
			y dir=both,
			y explicit,
			]
			table[x=x,y=y,y error=yerror]
			{
				x        y       		   yerror
				0.00340     0.90          0.05
				0.00330     0.91       	0.03
				0.00319   	0.86 	0.06
				0.00309  	0.8  	0.06      
			};
		
		\addplot[domain = 0.003:0.00340]{337.3383*x-0.2272};
			
		\end{axis}
	\end{tikzpicture}\\
	График 2. Зависимость $\mu$ от $\frac{1}{T}$  
\end{center}


\noindent Табличные коэффициенты газа Ван-дер-Ваальса для углекислого газа:\\
	\begin{equation}
		\begin{aligned}
				a = 0.36 \frac{\text{Па} \cdot \text{м}^6}{\text{моль}^2}\\
				b = 4.2 \cdot 10^{-5} \frac{\text{м}^3}{\text{моль}}
		\end{aligned}
	\end{equation}
  

	
	
	\section{Выводы}
	\begin{itemize}
		\item Полученные коэффициенты Джоуля-Томпсона отличаются от табличных в интервале от 10-20 \%;
		\item Углекислый газ действительно охлаждался в ходе эксперимента. Такое поведение подтверждает теорию;
		\item Трение, которое возникает в пористой пробке, значительно влияет вначале эксперимента до момента установления температуры в трубке;
		\item Полученные при расчете коэффициенты газа Ван-дер-Вальса имееют огромную погрешность и не соответствуют теории. Получить адекватную оценку температуры инверсии таким способом для углекислого газа невозможно.
	\end{itemize}
	
\end{document}