\documentclass[a4paper, 12pt]{article}
\usepackage[english,russian]{babel}
\usepackage{amsfonts}
\usepackage{pgfplots}
\usepackage[utf8]{inputenc}
\usepackage{amsmath,amssymb}
\usepackage{textcomp,mathcomp}
\usepackage{pgfplots}

\begin{document}
	\begin{Huge}
		\begin{center}
			\textbf{Отчет о выполнении лабораторной работы 2.1.3}\\
			\vspace{2em}
			\textbf{Определение $C_p$/$C_v$ по скорости звука в газе}\\
			\vspace{5em}
			\textbf{Выполнил: }Тимонин Андрей\\
			\textbf{Группа: }Б01-208\\
			\vspace{9em}
			\textbf{Дата: }20.04.2023\\
		\end{center}
	\end{Huge}
	
	\section{Введение}
	\noindent\textbf{Цели работы:}
	\begin{enumerate}
		\item Измерение частоты колебаний и длины волны при резонансе звуковых колебаний в газе, заполняющем трубу;
		\item Определение показателя адиабаты с помощью уравнения состояния идеального газа.
	\end{enumerate}
	\bigskip
	
	\noindent\textbf{В работе используются:}
	\begin{enumerate}
		\item Звуковой генератор;
		\item Электронный осциллограф;
		\item Раздвижная труба;
		\item Теплоизолированная труба;
		\item Обогреваемая водой из термостата;
		\item Баллон со сжатым углекислым газом;
		\item Газгольдер;
	\end{enumerate}

	\section{Теоретические сведения}
	Скорость звука в газах определяется формулой:
	\begin{equation}
		c = \sqrt{\gamma \frac{RT}{\mu}}
	\end{equation}
	R - газовая постоянная, T -температура газа, $\mu$ - молярная масса, $\gamma$ - показатель адиабаты
	Тогда:
	\begin{equation}
		\gamma = \frac{\mu}{RT}c^2
	\end{equation} 
	Условие резонанса (амплитуда звуковых колебаний резко возрастает):
	\begin{equation}
		L = n\frac{\lambda}{2}
	\end{equation}
	Связь параметров волны:
	\begin{equation}
		c=\lambda f
	\end{equation}
	Подбор условий резонанса:
	1) f = const, $L\neq const$
	\begin{equation}
		L_{n+k} = n\frac{\lambda}{2} + k \frac{\lambda}{2}
	\end{equation}
	Тогда $\frac{\lambda}{2}$ - угловой коэффициент графика зависимости L от k.
	
	2) L = const, $f\neq const$
	\begin{equation}
		L=\frac{\lambda_1}{2}n=\frac{\lambda_2}{2}(n+1)=...=\frac{\lambda_{k+1}}{2}(n+k)
	\end{equation}
	Тогда:
	\begin{equation}
		f_{k+1} = f_1+\frac{c}{2L}k
	\end{equation}
	Тогда $c/2L$ - угловой коэффициент графика зависимости частоты от номера резонанса.

	\section{Экспериментальная установка}
	\begin{center}
		\includegraphics[width=10cm, height=6cm]{"Установка1.jpg"}\\
		Рис. 1 Установка для измерения скорости звука при помощи раздвижной трубки
		
		\includegraphics{"Установка2.jpg"}\\
		Рис. 2 Установка для изучения зависимости скорости звука от температуры
	\end{center}

\section{Данные измерений при постоянной длине трубы}
Длина трубы $L = 80.0 \pm 0.1$ см
\begin{center}
	\begin{tabular} {|c | c |c |c |}
		\hline	
		$20 \tccelsius$ & $40 \tccelsius$ & $50 \tccelsius$ & $60 \tccelsius$\\
		\hline
		1.52 кГц &  1.70 кГц & 2.40 кГц & 2.50 кГц\\
		\hline
		1.73 кГц &  2.40 кГц & 3.30 кГц & 2.70 кГц\\
		\hline
		3.00 кГц &  3.10 кГц & 3.60 кГц & 2.98 кГц\\
		\hline
		3.46 кГц &  3.30 кГц & 3.84 кГц & 3.21 кГц\\
		\hline
		&  3.70 кГц &  & 3.44 кГц\\
		\hline
	\end{tabular}\\
	Таблица 1. Результаты измерений при постоянной длине трубы
\end{center}
	Угловой коэффициент графика равен:
	\begin{center}
		\begin{equation}
			\frac{f}{n} = \frac{c}{2\cdot L}
		\end{equation}
	\end{center}


	\begin{tikzpicture}
	\begin{axis}
		[
		width=13cm,
		mark=square,
		grid=major,
		xlabel={k},
		ylabel={$f - f_1$, Гц},
		]
		
		\addplot+[
		only marks,
		error bars/.cd,
		y dir=both,
		y explicit,
		]
		table[x=x,y=y,y error=yerror]
		{
			x        y       		   yerror
			1     0            0
			2     210           0
			3     1480         	 0
			4   	1940            0
		};
	
	\addplot+[
	only marks,
	error bars/.cd,
	y dir=both,
	y explicit,
	]
	table[x=x,y=y,y error=yerror]
	{
		x        y       		   yerror
		1     0             0
		2     700           0
		3     1400         	 0
		4   	1600            0
		5   	2000           0
	};

	\addplot+[
	only marks,
	error bars/.cd,
	y dir=both,
	y explicit,
	]
	table[x=x,y=y,y error=yerror]
	{
		x        y       		   yerror
		1     0             0
		2    900           0
		3     1200        	 0
		4   	1440           0
	};

	\addplot+[
	only marks,
	error bars/.cd,
	y dir=both,
	y explicit,
	]
	table[x=x,y=y,y error=yerror]
	{
		x        y       		   yerror
		1     0             0
		2    200           0
		3     480       	 0
		4   	710           0
		5   	940          0
	};
		
	\legend{20$\tccelsius$, 40$\tccelsius$, 50$\tccelsius$, 60 $\tccelsius$}	
	\addplot[domain = 0:6, color= blue]{709*x-865};
	\addplot[domain = 0:6, color= red]{490*x-330};
	\addplot[domain = 0:6, color= brown]{462*x-111.1};
	\addplot[domain = 0:6]{239*x-251};
	
	\end{axis}
\end{tikzpicture}\\
\begin{center}
	Рис 3. Графики измерений при постоянной длине трубы
\end{center}

\begin{center}
	\begin{tabular} {|c | c |c |}
		\hline	
		$c, \frac{\textbf{м}}{\textbf{с}}$ & $\bigtriangleup_c, \frac{\text{м}}{\text{с}}$, & T, $\tccelsius$\\
		\hline
		1134.4& 138.0& 20\\
		\hline
		784.0& 60.8& 40\\
\hline
		739.2& 111.1& 50\\
\hline
		382.4& 6.1& 60\\
\hline

	\end{tabular}\\
	Таблица 2. Скорости звука по коэффициентам наклона графиков
\end{center}

\textbf{Замечание:} результаты измерений для 20, 40 и 50 градусов не соответствуют теоритическим данным. Такие плохие результаты могут быть вызваны неправильным определением точек резонанса, недостаточным количеством экспериментальных точек или ошибкой аппроксимации МНК. Скорость должна зависеть от температуры, графики при это должны иметь приблизительно одинаковые коэффициенты наклонна $\approx 350 $

Проверим сходство результата для 60 $\tccelsius$ ниже:
\begin{center}
	\begin{equation}
		c = \sqrt{\gamma \cdot \frac{R\cdot T}{\mu}} = \sqrt{1.4 \cdot \frac{8.31 \cdot (273 + 60)}{0.029}} = 365.5
	\end{equation}
\end{center}

Ошибка составляет: $\frac{382.4 - 365.5}{365.5} = 0.05 \approx 5 \%$ - это достаточно точный результат.

Найдем показатель адиабаты при температуре 60 $\tccelsius$:
	\begin{center}
		\begin{equation}
			\gamma_{\text{воздух}} = \sqrt{\frac{c^2 \cdot \mu}{R \cdot T}} = \sqrt{\frac{382.4^2 \cdot 0.029}{8.31 \cdot (273+60)}} = 1.24
		\end{equation}
	\end{center}

Найдем ошибку определения показателя адиабаты при 60 $\tccelsius$: $\frac{|1.24 - 1.4|}{1.4} = 0.11 \approx 11\%$
	
	
	\section{Данные измерений при постоянной температуре}
	\subsection[O2]{Данные для воздуха}

	\begin{center}
		\begin{tabular} {|c | c |}
			\hline	
			№ & $\bigtriangleup_L$, см\\
			\hline
			1& 0.0\\
			\hline
			 2& 1.1\\
			\hline
		 	 3& 2.2\\
			\hline
			 4& 3.2\\
			\hline
			 5& 4.1\\
			\hline
			 6& 5.2\\
			\hline
		\end{tabular}\\
		Таблица 3. Результаты измерение при постоянной частоте 1.68 кГц
	\end{center}

	\begin{center}
		\begin{tabular} {|c | c |}
			\hline	
			№ & $\bigtriangleup_L$, см\\
			\hline
			1& 0.0\\
			\hline
			2& 7.0\\
			\hline
			3& 13.5\\
			\hline
			4& 19.6\\
			\hline
		\end{tabular}\\
		Таблица 4. Результаты измерение при постоянной частоте 2.72 кГц
	\end{center}

	\begin{center}
		\begin{tabular} {|c | c |}
			\hline	
			№ & $\bigtriangleup_L$, см\\
			\hline
			1& 0.0\\
			\hline
			2& 6.1\\
			\hline
			3& 11.9\\
			\hline
			4& 18.6\\
			\hline
			5& 23.3\\
			\hline
		\end{tabular}\\
		Таблица 5. Результаты измерение при постоянной частоте 3.02 кГц
	\end{center}

	\begin{center}
		\begin{tabular} {|c | c |}
			\hline	
			№ & $\bigtriangleup_L$, см\\
			\hline
			1& 0.0\\
			\hline
			2& 2.8\\
			\hline
			3& 8.4\\
			\hline
			4& 13.3\\
			\hline
			5& 19.5\\
			\hline
		\end{tabular}\\
		Таблица 6. Результаты измерение при постоянной частоте 3.19 кГц
	\end{center}

	\subsection[Carbon]{Данные для углекислого газа}
	
	\begin{center}
		\begin{tabular} {|c | c |}
			\hline	
			№ & $\bigtriangleup_L$, см\\
			\hline
			1& 0.0\\
			\hline
			2& 4.1\\
			\hline
			3& 9.1\\
			\hline
			4& 14.2\\
			\hline
			5& 19.7\\
			\hline
		\end{tabular}\\
		Таблица 7. Результаты измерение при постоянной частоте 2.63 кГц
	\end{center}

	\begin{center}
		\begin{tabular} {|c | c |}
			\hline	
			№ & $\bigtriangleup_L$, см\\
			\hline
			1& 0.0\\
			\hline
			2& 8.8\\
			\hline
			3& 13.6\\
			\hline
			4& 23.3\\
			\hline
		\end{tabular}\\
	Таблица 8. Результаты измерение при постоянной частоте 10.0 кГц
	\end{center}

	\begin{center}
		\begin{tabular} {|c | c |}
		\hline	
		№ & $\bigtriangleup_L$, см\\
		\hline
		1& 0.0\\
		\hline
		2& 11.1\\
		\hline
		3& 18.2\\
		\hline
		\end{tabular}\\
		Таблица 9. Результаты измерение при постоянной частоте 12.0 кГц
	\end{center}

	\begin{center}
		\begin{tabular} {|c | c |}
		\hline	
		№ & $\bigtriangleup_L$, см\\
		\hline
		1& 0.0\\
		\hline
		2& 8.2\\
		\hline
		3& 16.0\\
		\hline
		\end{tabular}\\
	Таблица 10. Результаты измерение при постоянной частоте 22.6 кГц
	\end{center}


	\subsection[Установка]{Построение графиков}
	\begin{tikzpicture}
		\begin{axis}
			[
			width=13cm,
			mark=square,
			grid=major,
			xlabel={k},
			ylabel={$\bigtriangleup_L$, см},
			]
			
			\addplot+[
			only marks,
			error bars/.cd,
			y dir=both,
			y explicit,
			]
			table[x=x,y=y,y error=yerror]
			{
				x        y       		   yerror
				1     0.0            0.1
				2     4.1           0.1
				3     9.1         	 0.1
				4   	14.2           0.1
			   	5  		19.7           0.1
			};
			
			\addplot+[
			only marks,
			error bars/.cd,
			y dir=both,
			y explicit,
			]
			table[x=x,y=y,y error=yerror]
			{
				x        y       		   yerror
				1     0.0            0.1
				2    8.8           0.1
				3    13.6         	 0.1
				4   	23.3          0.1
			};
		
			\addplot+[
			only marks,
			error bars/.cd,
			y dir=both,
			y explicit,
			]
			table[x=x,y=y,y error=yerror]
			{
				x        y       		   yerror
				1     0.0            0.1
				2    11           0.1
				3     18.2         	 0.1
			};
		
			\addplot+[
			only marks,
			error bars/.cd,
			y dir=both,
			y explicit,
			]
			table[x=x,y=y,y error=yerror]
			{
				x        y       		   yerror
				1     0.0            0.1
				2    8.2           0.1
				3     16.0        	 0.1
			};
		
		\legend{2.63 кГц, 10.00 кГц,12.00 кГц, 22.60 кГц}
		
		\addplot[domain = 0:6, color= blue]{4.95*x-5.43};
		\addplot[domain = 0:6, color= red ]{7.47*x-7.25};
		\addplot[domain = 0:6, color= brown ]{9.1*x-8.47};
		\addplot[domain = 0:6]{8*x-7.93};
			
		\end{axis}
	\end{tikzpicture}\\
	\begin{center}
		Рис 4. Графики для углекислого газа
	\end{center}

	\begin{tikzpicture}
	\begin{axis}
		[
		width=13cm,
		mark=square,
		grid=major,
			xlabel={k},
ylabel={$\bigtriangleup_L$, см},
		]
		
		\addplot+[
		only marks,
		error bars/.cd,
		y dir=both,
		y explicit,
		]
		table[x=x,y=y,y error=yerror]
		{
			x        y       		   yerror
			1     0.0             0.1
			2     7.0            0.1
			3     13.5         	  0.1
			4   	19.6            0.1
		};
		
		\addplot+[
		only marks,
		error bars/.cd,
		y dir=both,
		y explicit,
		]
		table[x=x,y=y,y error=yerror]
		{
			x        y       		   yerror
			1     0.0             0.1
			2    6.1            0.1
			3     11.9         	  0.1
			4   	18.6            0.1
			5   	23.3            0.1
		};
		
		\addplot+[
		only marks,
		error bars/.cd,
		y dir=both,
		y explicit,
		]
		table[x=x,y=y,y error=yerror]
		{
			x        y       		   yerror
			1     0.0            0.1
			2    2.8            0.1
			3     8.4         	 0.1
			4   	13.3            0.1
			5   	19.5            0.1
		};
		
		\addplot+[
		only marks,
		error bars/.cd,
		y dir=both,
		y explicit,
		]
		table[x=x,y=y,y error=yerror]
		{
			x        y       		   yerror
			1     0.0            0.1
			2   1.1            0.1
			3     2.2         	  0.1
			4   	3.2            0.1
			5   	4.1            0.1
			6   	5.2            0.1
		};
	
		\legend{2.72 кГц, 3.02 кГц, 3.19 кГц, 1.68 кГц}
	
	\addplot[domain = 0:7]{1.028*x-0.96};
	\addplot[domain = 0:7, color= blue]{6.52*x-6.3};
	\addplot[domain = 0:7, color= red]{5.91*x-5.75};
	\addplot[domain = 0:7, color= brown]{4.95*x-6.05};
	
	
		\end{axis}
	\end{tikzpicture}\\
	\begin{center}
		Рис 5. Графики для кислорода
	\end{center}

	\begin{center}
	\begin{tabular} {|c | c |c |}
		\hline	
		$\frac{\lambda}{2}$, см & f, кГц & $\bigtriangleup_{\lambda}$, см \\
		\hline
		9.10& 2.63& 1.10\\
		\hline
		7.47& 10.00& 0.65\\
		\hline
		4.95& 12.00& 0.15\\
		\hline
		8.00& 22.60& 0.12\\
		\hline
		
	\end{tabular}\\
	Таблица 11. Коэффициенты наклона графиков для углекислого газа
\end{center}

\begin{center}
	\begin{tabular} {|c | c |c |}
		\hline	
		$\frac{\lambda}{2}$, см & f, кГц & $\bigtriangleup_{\lambda}$, см \\
		\hline
		1.03& 1.68& 0.02\\
		\hline
		6.52& 2.72& 0.14\\
		\hline
		5.91& 3.02& 0.16\\
		\hline
		4.95& 3.19& 0.33\\
		\hline
		
	\end{tabular}\\
	Таблица 12. Коэффициенты наклона графиков для кислорода 
\end{center}

	Получаем показатель адиабаты для воздуха:
	$\gamma_{\text{воздух}} = 1.52 \pm 0.36 $

	
	\section{Выводы}
	\begin{itemize}
		\item Первое измерение резонанса при 1.68 кГц ошибочно и не соответствует теории. Ошибка может быть связана с неправильным определением точек резонанса в трубе;
		\item Гамма для воздуха действительно достаточно точно соответствует табличному значению $1.52\approx 1.403$ (табличное значение);
		\item Погрешность измерения показателя адиабаты сильно привязана к погрешности определения удлинения трубы;
		\item При больших частотах нельзя получить коэффициент адиабаты углекислого газа, это хорошо видно из расчетов;
		\item При измерении на установке №2 необходимо плавно увеличивать частоту, увеличить количество экспериментальных точек для избежания неправильных результатов вычисления скорости звука;
		\item Несмотря на ошибки, которые мы допустили при работе на установке №2, нами был получен показатель адиабаты для воздуха при 60 $\tccelsius$ с достаточно хорошей точностью. $\gamma_{\text{воздух}} = 1.24 \approx 1.402$(табличное значение).
\end{itemize}
	
\end{document}