\documentclass[a4paper, 12pt]{article}
\usepackage[english,russian]{babel}
\usepackage{amsfonts}
\usepackage{pgfplots}
\usepackage[utf8]{inputenc}
\usepackage{amsmath,amssymb}
\usepackage{textcomp,mathcomp}
\usepackage{pgfplots}

\begin{document}
	\begin{Huge}
		\begin{center}
			\textbf{Отчет о выполнении лабораторной работы 2.1.4}\\
			\vspace{2em}
			\textbf{Определение теплоемкости твердых тел}\\
			\vspace{5em}
			\textbf{Выполнил: }Тимонин Андрей\\
			\textbf{Группа: }Б01-208\\
			\vspace{9em}
			\textbf{Дата: }20.04.2023\\
		\end{center}
	\end{Huge}
	
	\section{Введение}
	\noindent\textbf{Цели работы:}
	\begin{enumerate}
		\item Измерение количества подведенного тепла и вызванного им нагрева твердого тела;
		\item Определение теплоемкости по экстраполяции отношения $\frac{\bigtriangleup Q}{\bigtriangleup T}$ к нулевым потерям тепла.
	\end{enumerate}
	\bigskip
	
	\noindent\textbf{В работе используются:}
	\begin{enumerate}
		\item Калориметр с нагревателем и термометром сопротивления;
		\item Амперметр;
		\item Вольтметр;
		\item Мост постоянного тока;
		\item Источник питания 36 В.
	\end{enumerate}
	
	\section{Теоретические сведения}
В данной работе теплоемкость определяется по формуле
\begin{equation}
	C = \frac{\Delta Q}{\Delta T},
	\label{eq:dQdT}
\end{equation}

где $\Delta Q$ -- количество тепла, подведенного к телу, и $\Delta T$ -- изменение температуры тела, произошедшее в результате подвода тепла.

Температура исследуемого тела надежно измеряется термометром сопротивления, а определение количества тепла, поглощенного телом, обычно вызывает затруднение. В реальных условиях не вся энергия $P \Delta t$, выделенная нагревателем, идет на нагревание исследуемого тела и калориметра, часть ее уходит из калориметра благодаря теплопроводности его стенок. Оставшееся в калориметре количество тепла $\Delta Q$ равно 
\begin{equation}
	\Delta Q = P\Delta t - \lambda(T - T_{\text{к}}) \Delta t,
	\label{eq:dQ}
\end{equation}
где $P$ -- мощность нагревателя, $\lambda$ -- коэффициент теплоотдачи стенок, $T$ -- температура тела, $T_{\text{к}}$ -- комнатная температура, $ \Delta t$ -- время, в течение которого идет нагревание.

Из уравнений (1) и (2) получаем
\begin{equation}
	C = \frac{P - \lambda(T - T_{\text{к}})}{\Delta T / \Delta t}
	\label{osnovnaya}
\end{equation}
Формула (3) является основной расчетной формулой. Она определяет теплоемкость тела вместе с калориметром. Теплоемкость калориметра измеряется отдельно и вычитается из результата.

С увеличением температуры исследуемого тела растет утечка энергии, связанная с теплопроводностью стенок калориметра. Из формулы (2) видно что при постоянной мощности нагревателя по мере роста температуры количество тепла передаваемое телу, уменьшается, и, следовательно, понижается скорость изменения его температуры.

Погрешности, связанные с утечкой тепла, оказываются небольшими, если не давать телу заметных перегревов и проводить все измерения при температурах, мало отличающихся от комнатной. Однако при небольших перегревах возникает большая ошибка при измерении $\Delta T = T - T_\text{к}$, и точность определения теплоемкости не возрастает. Чтобы избежать этой трудности, в работе используется следующая методика измерений. Зависимость скорости нагревания тела $\Delta T / \Delta t$ от температуры измеряется в широком интервале изменения температур. По полученным данным строится график
\begin{equation*}
	\frac{\Delta T}{\Delta t} = f(T).
\end{equation*}
Этот график экстраполируется к температуре $T = T_{\text{к}}$, и таким образом определяется скорость нагревания при комнатной температуре $(\Delta T / \Delta t)_{T_{\text{к}}}$. Подставляя полученное выражение в формулу (3) и замечая, что при $T = T_{\text{к}}$ член $\lambda(T - T_{\text{к}})$ обращается в ноль, получаем
\begin{equation}
	C = \frac{P}{(\Delta T / \Delta t)_{T_{\text{к}}}}
	\label{4}
\end{equation}

Температура измеряется термометром сопротивления, который представляет собой медную проволоку, намотанную на теплопроводящий каркас внутренней стенки калориметра (рис. 1). Сопротивление проводника изменяется с температурой по закону

\begin{equation}
	R_{T} = R_{0}(1 + \alpha \Delta T),
	\label{RT}
\end{equation}

где $R_{T}$ -- сопротивление термеметра про $T  ^{\circ}C$, $R_{0}$ -- его сопротивление при $0  ^{\circ}C$, $\alpha$ -- температурный коэффициент сопротивления. 

Дифференцируя (5) по времени, найдем

\begin{equation}
	\frac{dR}{dt} = R_{0}\alpha \frac{dT}{dt},
	\label{dRT}
\end{equation}

Выразим сопротивление $R_{0}$ через исмеренное значение $R_{\text{к}}$ -- сопротивление термометра при комнатной температуре. Согласно (5), имеем

\begin{equation}
	R_{0} = \frac{R_{\text{к}}}{1 + \alpha \Delta T_{\text{к}}},
	\label{R0}
\end{equation}

Подставляя (6) и (7) в (4), найдем

\begin{equation}
	C = \frac{PR_{\text{к}} \alpha}{(\frac{dR}{dt})_{T_{\text{к}}}(1 + \alpha \Delta T_{\text{к}})},
	\label{capacity}
\end{equation}

Входящий в формулу температурный коэффициент сопротивления меди равен $\alpha = 4,28 \cdot 10^{-3}~\text{град}^{-1}$, все остальные величины определяются экспериментально. 
	
	\section{Экспериментальная установка}
	Установка состоит из калориметра с пенопластовой изоляцией, помещенного в ящик из многослойной клееной фанеры. Внутренние стенки калориметра выполненым из материала с высокой теплопроводностью. Надежность теплового контакта между телом и стенками обеспечивается их формой: они имеют вид усеченных конусов и плотно прилегают друг к другу. В стенку калориметра вмонтированы электронагреватель и термометр сопротивления. Схема включения нагревателя изображения на рис.2. Система реостатов позволяет установить нужную силу тока в цепи нагревателя. По амперметру и вольтметру определяется мощность, выделяемая в нагревателе. Величина сопротивления термометра измеряется мостом постоянного тока.
	
	\begin{center}
		\includegraphics[width= 15cm]{"Установка1.jpg"}\\
		Рис 1. Схема калориметра
	\end{center}
	
	
	\section{Ход работы}
	Начальные данные:	
	
 	\begin{center}
 		\includegraphics[width=15cm]{"График1.jpg"}\\
 		Рис 2. График изменения комнатной температуры 
 		\includegraphics[width=15cm]{"График2.jpg"}\\
 		Рис 3. График изменений показаний омметра 
 		\includegraphics[width=15cm]{"Нагрев пустого.jpg"}\\
 		Рис 4. График нагрева пустого калориметра

		\includegraphics[width=15cm]{"Нагрев_алюм.jpg"}\\
		Рис 5. График нагрева калориметра с алюминиевым конусом
		
		\includegraphics[width=15cm]{"Нагрев_алюм.jpg"}\\
		Рис 6. График нагрева калориметра с железным конусом
		\includegraphics[width=15cm]{"Охлаждение пустого.jpg"}\\
		Рис 7.  График охлаждения пустого калориметра
		\includegraphics[width=15cm]{"Охлаждение железо.jpg"}\\
		Рис 8. График охлаждения калориметра с железным конусом
		\includegraphics[width=15cm]{"Охлаждение алюм.jpg"}\\
		Рис 9. График охлаждения калориметра с алюминиевым конусом
 	\end{center}
 
 
\begin{center}
 	\begin{tabular} {|c | c |}
		№ & $\frac{dR}{dt}$ \\
		\hline
		1 & 0.00152993 \\
		\hline
		2 & 0.00148243 \\
		\hline
		3 & 0.00150742 \\
		\hline
		4 & 0.00139527 \\
		\hline
		5 & 0.00136687 \\
		\hline
		6 & 0.00134701 \\
		\hline
		7 & 0.00133051 \\
		\hline
		8 & 0.00126687 \\
		\hline
 	\end{tabular}\\
 	Таблица 1. Значения $\frac{dR}{dt}$ для пустого калориметра
 \end{center}

\begin{center}
	\begin{tabular} {|c | c |}
		\hline	
		№ & $\frac{dR}{dt}$ \\
		\hline
		1 & 0.00096340 \\
		\hline
		2 & 0.00095203 \\
		\hline
		3 & 0.00091032 \\
		\hline
		4 & 0.00089963 \\
		\hline
		5 & 0.00088850 \\
		\hline
		6 & 0.00086551 \\
		\hline
		7 & 0.00087666 \\
		\hline
		8 & 0.00088192 \\
		\hline
	\end{tabular}\\
	Таблица 2. Значения $\frac{dR}{dt}$ для калориметра с алюминиевым конусом
\end{center}

\begin{center}
	\begin{tabular} {|c | c |}
		\hline	
		№ & $\frac{dR}{dt}$ \\
		\hline
		1 & 0.00078839 \\
		\hline
		2 & 0.00079701 \\
		\hline
		3 & 0.00074330 \\
		\hline
		4 & 0.00078119 \\
		\hline
		5 & 0.00074826 \\
		\hline
		6 & 0.00075064 \\
		\hline
		7 & 0.00075375 \\
		\hline
		8 & 0.00075512 \\
		\hline
	\end{tabular}\\
	Таблица 3. Значения $\frac{dR}{dt}$ для калориметра с железным конусом
\end{center}


	\begin{center}
	\begin{tikzpicture}
		\begin{axis}
			[
			legend pos=north east,
			width=13cm,
			mark=square,
			grid=major,
			ylabel={$\frac{dR}{dt}$},
			xlabel={n},
			]
			
			\legend{Пустой калориметр, C алюм. конусом, С желез. конусом}
			
			\addplot+[
			only marks,
			error bars/.cd,
			y dir=both,
			y explicit,
			]
			table[x=x,y=y,y error=yerror]
			{
				x        y       		   yerror
				1  0.00152993 0

				2  0.00148243 0

				3  0.00150742 0

				4  0.00139527 0

				5  0.00136687 0

				6  0.00134701 0

				7  0.00133051 0

				8  0.00126687 0
			};
			
			\addplot+[
			only marks,
			error bars/.cd,
			y dir=both,
			y explicit,
			]
			table[x=x,y=y,y error=yerror]
			{
				x        y       		   yerror
				1  0.00096340 0 

				2  0.00095203  0

				3  0.00091032 0

				4  0.00089963 0

				5  0.00088850 0

				6  0.00086551 0

				7  0.00087666 0

				8  0.00088192 0
			};
		
				\addplot+[
			mark=diamond*,
		only marks,
		error bars/.cd,
		y dir=both,
		y explicit,
		]
		table[x=x,y=y,y error=yerror]
		{
			x        y       		   yerror
		1  0.00078839 0

		2  0.00079701 0

		3  0.00074330 0

		4  0.00078119 0

		5  0.00074826 0

		6  0.00075064 0

		7  0.00075375 0

		8  0.00075512 0
		};

		\addplot [
		domain=0:9,
		color=black,
		]
		{-0.00003703*x+0.00156993};
		
		\addplot [
		domain=0:9,
		color=black,
		]
		{-0.00001301*x+0.00096329};
		
		\addplot [
		domain=0:9,
		color=black,
		]
		{-0.00000548*x+0.00078936};
			

			
			
			
		\end{axis}
	\end{tikzpicture}\\
	График 10. График $\frac{dR}{dt}$ для различных комбинаций при нагреве
\end{center}

	
\end{document}