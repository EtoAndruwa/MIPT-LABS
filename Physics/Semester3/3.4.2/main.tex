\documentclass[a4paper]{article}
\usepackage[utf8]{inputenc}
\usepackage[russian]{babel}
\usepackage[T2]{fontenc}
\usepackage[warn]{mathtext}
\usepackage{graphicx}
\usepackage{amsmath}
\usepackage{floatflt}
\usepackage{tikz}
\usepackage{pgfplots}
\usepackage{ gensymb }
\usepackage{ upgreek }
\usepackage[left=20mm, top=20mm, right=20mm, bottom=20mm, footskip=10mm]{geometry}


\graphicspath{ {images/} }
\usepackage{multicol}
\setlength{\columnsep}{2cm}


\begin{document}

\begin{titlepage}
	\centering
	\vspace{5cm}
	\vspace{4cm}
	{\scshape\Large Лабораторная работа 3.4.2\par}
	\vspace{1cm}
	{\huge\bfseries Закон Кюри-Вейсса\par}
	\vspace{1cm}
	\vfill
\begin{flushright}
	\vspace{0.3cm}
	{\LARGE Выполнил: Тимонин Андрей}
\end{flushright}
	

	\vfill

% Bottom of the page
	Долгопрудный, 2023 г.
\end{titlepage}

\section{Цель работы}

Изучение температурной зависимости магнитной восприимчивости ферромагнетика выше точки Кюри.

\section{В работе используются:}
\begin{itemize}
    \item катушка самоиндуктивности с образцом из гадолиния;
    \item термостат;
    \item частотомер;
    \item цифровой вольтметр;
    \item LC-автогенератор;
    \item термопара медь-константан;
\end{itemize}

\section{Ход работы}

\begin{table}[h]
    \centering
        \begin{tabular}{|c|c|c|c|}
        \cline{1-4}
        № & $T, \celsius$ & $\Delta V, \text{В}$ & $\tau, \text{мкс}$\\ \cline{1-4}
        1 & $14.13 \pm 0.01$ & $-0.000010 \pm 0.000001$& $10.065 \pm 0.001$\\ \cline{1-4}
        2 & $16.03 \pm 0.01$& $-0.000011 \pm 0.000001$ & $9.956 \pm 0.001$\\ \cline{1-4}
        3 & $18.03 \pm 0.01$& $-0.000012 \pm 0.000001$ & $9.711 \pm 0.001$\\ \cline{1-4}
        4 & $20.02 \pm 0.01$& $-0.000013 \pm 0.000001$ & $9.410 \pm 0.001$\\ \cline{1-4}
        5 & $22.01 \pm 0.01$& $-0.000018 \pm 0.000001$ & $9.046 \pm 0.001$\\ \cline{1-4}
        6 & $24.01 \pm 0.01$& $-0.000023 \pm 0.000001$ & $8.760 \pm 0.001$\\ \cline{1-4}
        7 & $26.01 \pm 0.01$& $-0.000012 \pm 0.000001$ & $8.602 \pm 0.001$\\ \cline{1-4}
        8 & $28.00 \pm 0.01$& $-0.000019 \pm 0.000001$ & $8.536 \pm 0.001$\\ \cline{1-4}
        9 & $30.00 \pm 0.01$& $-0.000018 \pm 0.000001$ & $8.487 \pm 0.001$\\ \cline{1-4}
        10 & $32.00 \pm 0.01$& $-0.000020 \pm 0.000001$ & $8.454 \pm 0.001$\\ \cline{1-4}
        11 & $34.00 \pm 0.01$& $-0.000019 \pm 0.000001$ & $8.428 \pm 0.001$\\ \cline{1-4}
        12 & $36.00 \pm 0.01$& $-0.000019 \pm 0.000001$ & $8.411 \pm 0.001$\\ \cline{1-4}
        13 & $38.00 \pm 0.01$& $-0.000019 \pm 0.000001$ & $8.395 \pm 0.001$\\ \cline{1-4}
        14 & $40.00 \pm 0.01$& $-0.000020 \pm 0.000001$ & $8.383 \pm 0.001$\\ \cline{1-4}
        
        \end{tabular}
        \caption{Данные эксперимента}
        \label{tab:my_label}
    \end{table}

\begin{equation}
    \Delta U_{\text{допустимая}} = \frac{0.5}{24} = 0.000021 \text{В}
\end{equation}

\textbf{ЗАМЕЧАНИЕ:} Необходимо учесть разность температур между водой и образцом используя показания термопары.

\begin{equation}
    \Delta T_{\text{поправка}} = 24000 \frac{\celsius}{\text{В}} \cdot \Delta V
\end{equation}

    \begin{table}[h]
    \centering
        \begin{tabular}{|c|c|c|}
        \cline{1-3}
        № & $\Delta T_{\text{поправка}}, \celsius$ & $T_{\text{итоговая}}, \celsius$\\ \cline{1-3}
        1 & $-0.240 \pm 0.024$ & $13.890 \pm 0.034$\\ \cline{1-3}
        2 & $-0.264 \pm 0.024$ & $15.766 \pm 0.034$\\ \cline{1-3}
        3 & $-0.288 \pm 0.024$ & $17.742 \pm 0.034$\\ \cline{1-3}
        4 & $-0.312 \pm 0.024$ & $19.708 \pm 0.034$\\ \cline{1-3}
        5 & $-0.432 \pm 0.024$ & $21.578 \pm 0.034$\\ \cline{1-3}
        6 & $-0.552 \pm 0.024$ & $23.458 \pm 0.034$\\ \cline{1-3}
        7 & $-0.288 \pm 0.024$ & $25.722 \pm 0.034$\\ \cline{1-3}
        8 & $-0.456 \pm 0.024$ & $27.544 \pm 0.034$\\ \cline{1-3}
        9 & $-0.432 \pm 0.024$ & $29.568 \pm 0.034$\\ \cline{1-3}
        10 & $-0.480 \pm 0.024$ & $31.520 \pm 0.034$\\ \cline{1-3}
        11 & $-0.456 \pm 0.024$ & $33.544 \pm 0.034$\\ \cline{1-3}
        12 & $-0.456 \pm 0.024$ & $35.544 \pm 0.034$\\ \cline{1-3}
        13 & $-0.456 \pm 0.024$ & $37.544 \pm 0.034$\\ \cline{1-3}
        14 & $-0.480 \pm 0.024$ & $39.520 \pm 0.034$\\ \cline{1-3}
        
        \end{tabular}
        \caption{Поправки к температурам образца и итоговые температуры}
        \label{tab:my_label}
    \end{table}

    \begin{table}[h]
    \centering
        \begin{tabular}{|c|c|c|}
        \cline{1-2}
        № &  $\tau^2 - \tau_0^2, \text{мкс}^2$\\ \cline{1-2}
        1 &  $33.209 \pm 0.037$\\ \cline{1-2}
        2 &  $31.026 \pm 0.036$ \\ \cline{1-2}
        3 &  $26.208 \pm 0.036$  \\ \cline{1-2}
        4 &  $20.453 \pm 0.035$  \\ \cline{1-2}
        5 &  $13.717 \pm 0.035$  \\ \cline{1-2}
        6 &  $8.625 \pm 0.034$  \\ \cline{1-2}
        7 &  $5.899 \pm 0.034$  \\ \cline{1-2}
        8 &  $4.768 \pm 0.034$  \\ \cline{1-2}
        9 &  $3.934 \pm 0.033$  \\ \cline{1-2}
        10 & $3.375 \pm 0.033$  \\ \cline{1-2}
        11 &  $2.936 \pm 0.033$ \\ \cline{1-2}
        12 &  $2.649 \pm 0.033$ \\ \cline{1-2}
        13 &  $2.381 \pm 0.033$ \\ \cline{1-2}
        14 &  $2.179 \pm 0.033$ \\ \cline{1-2}
        
        \end{tabular}
        \caption{Данные для графика 1}
        \label{tab:my_label}
    \end{table}

    \begin{table}[h]
    \centering
        \begin{tabular}{|c|c|c|}
        \cline{1-2}
        № &  $\frac{1}{\tau^2 - \tau_0^2}, \text{мкс}^{-2}$\\ \cline{1-2}
        1 &  $0.030 \pm 0.001$ \\ \cline{1-2}
        2 &  $0.032 \pm 0.001$ \\ \cline{1-2}
        3 &  $0.038 \pm 0.001$ \\ \cline{1-2}
        4 &  $0.049 \pm 0.002$ \\ \cline{1-2}
        5 &  $0.073\pm 0.003$ \\ \cline{1-2}
        6 &  $0.116\pm 0.004$ \\ \cline{1-2}
        7 &  $0.170\pm 0.006$ \\ \cline{1-2}
        8 &  $0.210\pm 0.007$ \\ \cline{1-2}
        9 &  $0.254\pm 0.009$ \\ \cline{1-2}
        10 & $0.296\pm 0.010$ \\ \cline{1-2}
        11 & $0.341\pm 0.011$ \\ \cline{1-2}
        12 & $0.377\pm 0.013$ \\ \cline{1-2}
        13 & $0.420\pm 0.014$ \\ \cline{1-2}
        14 & $0.459\pm 0.015$ \\ \cline{1-2}
        
        \end{tabular}
        \caption{Данные для графика 2}
        \label{tab:my_label}
    \end{table}

    

    \begin{tikzpicture}
    \begin{axis}
    [
        xlabel={$T, \celsius$}, 
        ylabel={$\tau^2 - \tau_0^2, \text{мкс}^2$}, 
        grid,
        legend pos=north east,
        height = 0.4\paperheight, 
	width = 0.8\paperwidth,
        title={График 1. Зависимость $\tau^2 - \tau_0^2$ от T},
    ]
    
    \addplot[blue, mark=x, only marks] plot [error bars/.cd, y dir = both, y explicit]
    table[x =x, y =y, y error =ey, x error =ex]{graph1.txt};\

    \addplot[red, domain=13:40]{-0.00000046574*x^7+0.00008699911*x^6-0.00678349062*x^5+0.28508966918*x^4-6.94453675364*x^3+97.67512396676*x^2-734.77532661281*x+2321.79252167450};

    \addplot +[mark=none, black] coordinates {(22.705, 0) (22.705, 30)};
    \addplot +[mark=none, blue] coordinates {(13.89, 0) (13.89, 35)};
    \addplot +[mark=none, blue] coordinates {(31.52, 0) (31.52, 35)};

    \addlegendentry{Экспериментальные данные}
    \addlegendentry{Аппроксимация полиномом 5-й степени}
    
    \end{axis}
    \end{tikzpicture}

Уравнение полинома 5-ой степени:
    \begin{equation*}
    \begin{split}
        y = -0.00000046574\cdotx^7+0.00008699911\cdotx^6-0.00678349062\cdotx^5+0.28508966918\cdotx^4 -\\-6.94453675364\cdotx^3
        +97.67512396676\cdotx^2-734.77532661281\cdotx+2321.79252167450
    \end{split}
    \end{equation*}

    Точка Кюри для гадолиния лежит посередине отрезка $\uptheta_K = \frac{T{\text{макс}} - T{\text{мин}}}{2}$ 

    Точка Кюри для гадолиния:
    \begin{equation}
        \uptheta_K = \frac{31.52 + 13.89}{2} = 22.705 \pm 0.048 \celsius (295.855 \pm 0.048 K)
    \end{equation}

    Табличное значение точки Кюри для гадолиния: $\uptheta_K_{\text{теор}} = 18.85 \celsius (292 K)$ 

    Погрешность экспериментального значения: $\delta \uptheta_K = \frac{22.705 - 18.85}{18.85} \cdot 100 \% = 20.45\%$ 
    
     \begin{tikzpicture}
    \begin{axis}
    [
        xlabel={$T, \celsius$}, 
        ylabel={$\frac{1}{\tau^2 - \tau_0^2}, \text{мкс}^{-2}$}, 
        grid,
        legend pos=north west,
        height = 0.4\paperheight, 
	width = 0.8\paperwidth,
        legend pos=north west,
        title={График 2. Зависимость $\frac{1}{\tau^2 - \tau_0^2}$ от T},
    ]
    
    \addplot[red, mark=x, only marks] plot [error bars/.cd, y dir = both, y explicit]
    table[x =x, y =y, y error =ey, x error =ex]{graph2.txt};\

    \addplot[black, domain=17:40]{x*0.0215 -0.3849};
    \addplot[black, domain=14:20]{0};


    \addlegendentry{Экспериментальные данные}
    \addlegendentry{Аппроксимация для парамагнитной точки Кюри(0.0215$\cdot$x-0.3849)}

    
    \end{axis}
    \end{tikzpicture}

Погрешности аппроксимации для парамагнитной точки Кюри $\Delta k = 0.0002 \frac{1}{\text{мкс}^{2} \cdot \celsius}, \Delta b = 0.0073 \text{мкс}^{-2}$ 

Парамагнитная точка Кюри из графика 2 (пересечение с осью абцисс) $\uptheta _p = 17.90 \pm 0.51 \celsius (291.05 \pm 0.51 K)$


\end{document}