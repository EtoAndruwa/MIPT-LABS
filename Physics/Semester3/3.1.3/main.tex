\documentclass[a4paper]{article}
\usepackage[utf8]{inputenc}
\usepackage[russian]{babel}
\usepackage[T2]{fontenc}
\usepackage[warn]{mathtext}
\usepackage{graphicx}
\usepackage{amsmath}
\usepackage{floatflt}
\usepackage{tikz}
\usepackage{pgfplots}
\usepackage[left=20mm, top=20mm, right=20mm, bottom=20mm, footskip=10mm]{geometry}


\graphicspath{ {images/} }
\usepackage{multicol}
\setlength{\columnsep}{2cm}


\begin{document}

\begin{titlepage}
	\centering
	\vspace{5cm}
	\vspace{4cm}
	{\scshape\Large Лабораторная работа 3.1.3\par}
	\vspace{1cm}
	{\huge\bfseries Измерение магнитного поля Земли\par}
	\vspace{1cm}
	\vfill
\begin{flushright}
	\vspace{0.3cm}
	{\LARGE Выполнил: Тимонин Андрей}
\end{flushright}
	

	\vfill

% Bottom of the page
	Долгопрудный, 2023 г.
\end{titlepage}

\section{Цель работы}

Исследовать свойства постоянных неодимовых магнитов; измерить с их помощью горизонтальную и вертикальную составляющие индукции магнитного поля Земли и магнитное наклонение.

\section{В работе используются:}
\begin{itemize}
    \item неодимовые магниты; 
    \item тонкая нить для изготовления крутильного маятника; 
    \item медная проволока; 
    \item электронные весы; секундомер; 
    \item измеритель магнитной индукции; 
    \item штангенциркуль; 
    \item брусок, линейка и штатив из немагнитных материалов; 
    \item набор гирь и разновесов.
\end{itemize}

\section{Ход работы}

\subsection{Задание 1}
\subsubsection{Способ А}
Масса одного шарика:
\begin{center}
    \begin{equation}
        m = \frac{6.8400}{8} = (0.8550 \pm 0.0001) \textit{ г}
    \end{equation}    
\end{center}

Диаметр одного шарика измерим штангенциркулем:
\begin{center}
    \begin{equation}
         d = (6.0 \pm 0.1) \textit{ мм}
    \end{equation}    
\end{center}

$r_{max}$ с учетом радиуса двух шариков:
\begin{center}
    \begin{equation}
         r_{max} = (1.67 \pm 0.01) \textit{ cм}
    \end{equation}    
\end{center}

Отсюда найдем намагниченность $P_m$:
\begin{center}
    \begin{equation}
         P_m = \sqrt{\frac{m_{\textit{одного шарика}} \cdot g \cdot (r_{max}^4)}{6}} = 32.967 \pm 0.339
    \end{equation}    
\end{center}

Легко найдем намагниченность материала:
\begin{center}
    \begin{equation}
         p_m = \frac{P_m}{V} = \frac{32.967}{\frac{4\cdot \pi \cdot R^3}{3}} = 291.49 \pm 32.15
    \end{equation}    
\end{center}

Остаточная намагниченность:
\begin{center}
    \begin{equation}
         B_r =  4\pi p_m = 3662.97 \pm 32.15
    \end{equation}    
\end{center}

Найдем величину $B_p$ шарика на полюсе:
\begin{center}
    \begin{equation}
         B_p = \frac{2B_r}{3} = 2441.98 \pm 32.15
    \end{equation}    
    $B_{p-prac} = 2302 \pm 1 $
\end{center}

\subsubsection{Способ Б}
Общая масса груза при отрыве:
\begin{center}
    \begin{equation}
         М = 298.430 \pm 14.776
    \end{equation}    
\end{center}

Отсюда найдем $F_0$:
\begin{center}
    \begin{equation}
         F_0 = \frac{F}{1.08} = 270981.34 \pm 13416.95 \textit{ дин}
    \end{equation}    
\end{center}

Найдем $P_m$ и $p_m$:
\begin{center}
    \begin{equation}
         P_m = \sqrt{\frac{F_0\cdot d^4}{6}} = 76.5 \pm 8.9
    \end{equation}    
\end{center}

\begin{center}
    \begin{equation}
         p_m = 676.46 \pm 146.35
    \end{equation}    
\end{center}

\textbf{Вывод: способ А точнее способа Б, можем найти $B_p$ и сравнить с $B_{p-prac}$}

\subsection{Задание 2}

\begin{table}[h]
    \centering
        \begin{tabular}{|c|c|c|}
        \cline{1-3}
        n & N & $T, \textit{с}$\\ \cline{1-3}
        12 & 5 & 40.22\\ \cline{1-3}
        10 & 6 & 39.52\\ \cline{1-3}
        \end{tabular}
        \caption{Данные для пункта 15}
        \label{tab:my_label}
    \end{table}

    \begin{table}[h]
    \centering
        \begin{tabular}{|c|c|c|}
        \cline{1-3}
        n & N & $T, \textit{с}$\\ \cline{1-3}
        3 &  10 & 11.35\\ \cline{1-3}
        4 &  10 & 15.02\\ \cline{1-3}
        5 &  10 & 17.86\\ \cline{1-3}
        6 &  10 & 20.68\\ \cline{1-3}
        7 &  10 & 25.16\\ \cline{1-3}
        8 &  10 & 29.70\\ \cline{1-3}
        9 &  10 & 32.83\\ \cline{1-3}
        10 &  10 & 34.28\\ \cline{1-3}
        11 &  10 & 41.35\\ \cline{1-3}
        12 &  10 & 43.90 \\ \cline{1-3}
        \end{tabular}
        \caption{Данные для пункта 16}
        \label{tab:my_label}
    \end{table}

    \begin{tikzpicture}
    \begin{axis}
    [
        xlabel={$n$}, 
        ylabel={$T$}, 
        grid,
        legend pos=north west,
        height = 0.4\paperheight, 
	width = 0.8\paperwidth,
        title={График зависимости периода колебаний $T$ от количества магнитов $n$ в магнитной стрелке},
    ]
    
    \addplot[blue, mark=x, only marks] plot [error bars/.cd, y dir = both, y explicit]
    table[x =x, y =y, y error =ey, x error =ex]{T_ot_n.txt};\
    \addplot[red, domain=2:13]{x*0.36384848 - 0.00756};

    \addlegendentry{T(n)}
    
    \end{axis}
    \end{tikzpicture}

Из аппроксимации данных графиков функции $T(n) = k \cdot n$, где $k = 0.3638$ получаем:
\begin{center}
    \begin{equation}
         B_{hor} = \frac{\pi ^ 2 \cdot m \cdot d^2}{3\cdot k^2 \cdot P_m} = 0.2320 \pm 0.0102
    \end{equation}    
\end{center}

\subsection{Задание 3}

\begin{table}[h]
    \centering
        \begin{tabular}{|c|c|}
        \cline{1-2}
        n & $m_{\textit{груза}}, \textit{г}$\\ \cline{1-2}
        10 & 0.099\\ \cline{1-2}
        8 & 0.126\\ \cline{1-2}
        6 & 0.196\\ \cline{1-2}
        4 & 0.159\\ \cline{1-2}
        \end{tabular}
        \caption{Данные для пункта 23}
        \label{tab:my_label}
    \end{table}

\begin{tikzpicture}
    \begin{axis}
    [
        xlabel={$n$}, 
        ylabel={$M$}, 
        grid,
        legend pos=north west,
        height = 0.4\paperheight, 
	width = 0.8\paperwidth,
        title={График зависимости механического момента $М$ груза от количества магнитов $n$ в магнитной стрелке},
    ]
    
    \addplot[blue, mark=x, only marks] plot [error bars/.cd, y dir = both, y explicit]
    table[x =x, y =y, y error =ey, x error =ex]{M_ot_n.txt};\
    
    \addplot[red, domain=5:11]{x*0.588399 + 223.98};

    \addlegendentry{M(n)}
    
    \end{axis}
    \end{tikzpicture}

Полученные значения $B_hor$ и $B_vert$:

\begin{center}
    \begin{equation}
         B_{vert} = \frac{M_{mech}}{n\cdot P_m} = 0.9053 \pm 0.0752
    \end{equation}    
\end{center}

Из аппроксимации графиком:

\begin{center}
    \begin{equation}
         B_{hor} = \frac{A}{P_m} = \frac{0.588399}{32.967} = 0.01784
    \end{equation}    
\end{center}

\begin{center}
    \begin{equation}
         B = \sqrt{(B_{hor})^2 + (B_{vert})^2} = 0.905541
    \end{equation}    
\end{center}

\begin{center}
    \begin{equation}
         cos\beta = \frac{B_{hor}}{B}
         \beta = 88.871
    \end{equation}    
\end{center}

\end{document}