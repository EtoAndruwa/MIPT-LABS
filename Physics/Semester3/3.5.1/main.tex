\documentclass[a4paper]{article}
\usepackage[utf8]{inputenc}
\usepackage[russian]{babel}
\usepackage[T2]{fontenc}
\usepackage[warn]{mathtext}
\usepackage{graphicx}
\usepackage{amsmath}
\usepackage{floatflt}
\usepackage{tikz}
\usepackage{pgfplots}
\usepackage{ gensymb }
\usepackage{ upgreek }
\usepackage[left=20mm, top=20mm, right=20mm, bottom=20mm, footskip=10mm]{geometry}


\graphicspath{ {images/} }
\usepackage{multicol}
\setlength{\columnsep}{2cm}


\begin{document}

\begin{titlepage}
	\centering
	\vspace{5cm}
	\vspace{4cm}
	{\scshape\Large Лабораторная работа 3.5.1\par}
	\vspace{1cm}
	{\huge\bfseries Изучение плазмы газового разряда в неоне\par}
	\vspace{1cm}
	\vfill
\begin{flushright}
	\vspace{0.3cm}
	{\LARGE Выполнил: Тимонин Андрей}
\end{flushright}
	

	\vfill

% Bottom of the page
	Долгопрудный, 2023 г.
\end{titlepage}

\section{Цель работы}

Изучение вольт-амперной характеристики тлеющего разряда; изучение свойств плазмы методом зондовых характеристик;

\section{В работе используются:}
\begin{itemize}
    \item стеклянная газоразрядная трубка, наполненная неоном;
    \item высоковольтный источник питания;
    \item источник питания постоянного тока;
    \item делитель напряжения;
    \item потенциометр;
    \item амперметры;
    \item вольтметры;
    \item переключатели;
\end{itemize}

\section{Ход работы}
\subsection{ВАХ}
\begin{table}[h]
    \centering
        \begin{tabular}{|c|c|c|}
        \cline{1-3}
        № & I, мА & V, В\\ \cline{1-3}
        1 & $0.5073 \pm 0.0001$& $34.350 \pm 0.001$ \\ \cline{1-3}
        2 & $1.0063 \pm 0.0001$& $32.394 \pm 0.001$\\ \cline{1-3}
        3 & $1.5088 \pm 0.0001$& $31.526 \pm 0.001$\\ \cline{1-3}
        4 & $2.0090 \pm 0.0001$& $26.864 \pm 0.001$\\ \cline{1-3}
        5 & $2.5017 \pm 0.0001$& $21.608 \pm 0.001$\\ \cline{1-3}
        6 & $3.0003 \pm 0.0001$& $18.850 \pm 0.001$\\ \cline{1-3}
        7 & $3.5057 \pm 0.0001$& $16.873 \pm 0.001$\\ \cline{1-3}
        8 & $4.0023 \pm 0.0001$& $16.085 \pm 0.001$\\ \cline{1-3}
        9 & $4.5049 \pm 0.0001$& $15.425 \pm 0.001$\\ \cline{1-3}
        10 & $5.0013 \pm 0.0001$& $14.715 \pm 0.001$\\ \cline{1-3}
        \end{tabular}
        \caption{Повышение тока разряда}
        \label{tab:my_label}
    \end{table}

\begin{table}[h]
    \centering
        \begin{tabular}{|c|c|c|}
        \cline{1-3}
        № &I, мА & V, В\\ \cline{1-3}
        1 & $4.9995 \pm 0.0001$& $14.696 \pm 0.001$ \\ \cline{1-3}
        2 & $4.5084 \pm 0.0001$& $15.367 \pm 0.001$\\ \cline{1-3}
        3 & $4.0085 \pm 0.0001$& $15.956 \pm 0.001$\\ \cline{1-3}
        4 & $3.508 \pm 0.0001$& $16.739 \pm 0.001$\\ \cline{1-3}
        5 & $2.996 \pm 0.0001$& $18.795 \pm 0.001$\\ \cline{1-3}
        6 & $2.5028 \pm 0.0001$& $21.492 \pm 0.001$\\ \cline{1-3}
        7 & $2.0073 \pm 0.0001$& $26.418 \pm 0.001$\\ \cline{1-3}
        8 & $1.5054 \pm 0.0001$& $31.496 \pm 0.001$\\ \cline{1-3}
        9 & $1.0007 \pm 0.0001$& $32.413 \pm 0.001$\\ \cline{1-3}
        10 & $0.501 \pm 0.0001$& $34.408 \pm 0.001$\\ \cline{1-3}
        \end{tabular}
        \caption{Понижение тока разряда}
        \label{tab:my_label}
    \end{table}

\subsection{Зондовые характеристики}
       \begin{table}[h]
    \centering
        \begin{tabular}{|c|c|c|}
        \cline{1-3}
        № & I, мкА& V, В\\ \cline{1-3}
        1 & $-61.85\pm 0.01$& $25.041 \pm 0.001$\\ \cline{1-3}
        2 & $-59.77\pm 0.01$& $22.045 \pm 0.001$\\ \cline{1-3}
        3 & $-57.05\pm 0.01$& $19.010 \pm 0.001$\\ \cline{1-3}
        4 & $-53.19\pm 0.01$& $16.031 \pm 0.001$\\ \cline{1-3}
        5 & $-45.80\pm 0.01$& $13.012 \pm 0.001$\\ \cline{1-3}
        6 & $-34.79\pm 0.01$& $10.031 \pm 0.001$\\ \cline{1-3}
        7 & $-24.78\pm 0.01$& $8.006 \pm 0.001$\\ \cline{1-3}
        8 & $-13.06\pm 0.01$& $6.003 \pm 0.001$\\ \cline{1-3}
        9 & $-0.47\pm 0.01$& $4.013 \pm 0.001$\\ \cline{1-3}
        10 & $14.70\pm 0.01$& $2.027 \pm 0.001$\\ \cline{1-3}
        11 & $24.36\pm 0.01$& $0.5146 \pm 0.001$\\ \cline{1-3}
        \end{tabular}
        \caption{Зондовые характеристики при $I_p = 5.0262$ мА и $\text{П}_2$ +}
        \label{tab:my_label}
    \end{table}

        \begin{table}[h]
    \centering
        \begin{tabular}{|c|c|c|}
        \cline{1-3}
        № & I, мкА& V, В\\ \cline{1-3}
        1 & $-80.97\pm 0.01$& $25.009 \pm 0.001$\\ \cline{1-3}
        2 & $-78.87\pm 0.01$& $22.014 \pm 0.001$\\ \cline{1-3}
        3 & $-76.15\pm 0.01$& $19.092 \pm 0.001$\\ \cline{1-3}
        4 & $-71.99\pm 0.01$& $16.094 \pm 0.001$\\ \cline{1-3}
        5 & $-65.51\pm 0.01$& $13.090 \pm 0.001$\\ \cline{1-3}
        6 & $-55.29\pm 0.01$& $10.059 \pm 0.001$\\ \cline{1-3}
        7 & $-46.03\pm 0.01$& $8.040 \pm 0.001$\\ \cline{1-3}
        8 & $-35.11\pm 0.01$& $6.072 \pm 0.001$\\ \cline{1-3}
        9 & $-21.91\pm 0.01$& $4.018 \pm 0.001$\\ \cline{1-3}
        10 & $-8.78\pm 0.01$& $2.013 \pm 0.001$\\ \cline{1-3}
        11 & $1.01\pm 0.01$& $0.576 \pm 0.001$\\ \cline{1-3}
        \end{tabular}
        \caption{Зондовые характеристики при $I_p = 5.0262$ мА и $\text{П}_2$ -}
        \label{tab:my_label}
    \end{table}

        \begin{table}[h]
    \centering
        \begin{tabular}{|c|c|c|}
        \cline{1-3}
        № & I, мкА& V, В\\ \cline{1-3}
        1 & $-45.04\pm 0.01$& $25.009 \pm 0.001$\\ \cline{1-3}
        2 & $-43.65\pm 0.01$& $22.097 \pm 0.001$\\ \cline{1-3}
        3 & $-42.13\pm 0.01$& $19.061 \pm 0.001$\\ \cline{1-3}
        4 & $-40.41\pm 0.01$& $16.085 \pm 0.001$\\ \cline{1-3}
        5 & $-37.59\pm 0.01$& $13.010 \pm 0.001$\\ \cline{1-3}
        6 & $-32.59\pm 0.01$& $10.033 \pm 0.001$\\ \cline{1-3}
        7 & $-27.23\pm 0.01$& $8.027 \pm 0.001$\\ \cline{1-3}
        8 & $-20.18\pm 0.01$& $6.035 \pm 0.001$\\ \cline{1-3}
        9 & $-10.91\pm 0.01$& $4.019 \pm 0.001$\\ \cline{1-3}
        10 & $-0.32\pm 0.01$& $2.053 \pm 0.001$\\ \cline{1-3}
        11 & $8.83\pm 0.01$& $0.474 \pm 0.001$\\ \cline{1-3}
        \end{tabular}
        \caption{Зондовые характеристики при $I_p = 3.0094$ мА и $\text{П}_2$ +}
        \label{tab:my_label}
    \end{table}

        \begin{table}[h]
    \centering
        \begin{tabular}{|c|c|c|}
        \cline{1-3}
        № & I, мкА& V, В\\ \cline{1-3}
        1 & $-61.12\pm 0.01$& $25.010 \pm 0.001$\\ \cline{1-3}
        2 & $-59.45\pm 0.01$& $22.050 \pm 0.001$\\ \cline{1-3}
        3 & $-57.73\pm 0.01$& $19.185 \pm 0.001$\\ \cline{1-3}
        4 & $-55.71\pm 0.01$& $16.103 \pm 0.001$\\ \cline{1-3}
        5 & $-52.96\pm 0.01$& $13.026 \pm 0.001$\\ \cline{1-3}
        6 & $-48.43\pm 0.01$& $10.019 \pm 0.001$\\ \cline{1-3}
        7 & $-43.83\pm 0.01$& $8.042 \pm 0.001$\\ \cline{1-3}
        8 & $-37.56\pm 0.01$& $6.084 \pm 0.001$\\ \cline{1-3}
        9 & $-28.82\pm 0.01$& $3.999 \pm 0.001$\\ \cline{1-3}
        10 & $-18.84\pm 0.01$& $2.048 \pm 0.001$\\ \cline{1-3}
        11 & $-10.47\pm 0.01$& $0.577 \pm 0.001$\\ \cline{1-3}
        \end{tabular}
        \caption{Зондовые характеристики при $I_p = 3.0094$ мА и $\text{П}_2$ -}
        \label{tab:my_label}
    \end{table}

        \begin{table}[h]
    \centering
        \begin{tabular}{|c|c|c|}
        \cline{1-3}
        № & I, мкА& V, В\\ \cline{1-3}
        1 & $-24.02\pm 0.01$& $25.010 \pm 0.001$\\ \cline{1-3}
        2 & $-23.40\pm 0.01$& $22.012 \pm 0.001$\\ \cline{1-3}
        3 & $-22.67\pm 0.01$& $19.143 \pm 0.001$\\ \cline{1-3}
        4 & $-21.88\pm 0.01$& $16.127 \pm 0.001$\\ \cline{1-3}
        5 & $-20.79\pm 0.01$& $13.064 \pm 0.001$\\ \cline{1-3}
        6 & $-18.56\pm 0.01$& $10.106 \pm 0.001$\\ \cline{1-3}
        7 & $-15.85\pm 0.01$& $8.066 \pm 0.001$\\ \cline{1-3}
        8 & $-11.95\pm 0.01$& $6.025 \pm 0.001$\\ \cline{1-3}
        9 & $-6.92\pm 0.01$& $4.0373 \pm 0.001$\\ \cline{1-3}
        10 & $-0.65\pm 0.01$& $2.016 \pm 0.001$\\ \cline{1-3}
        11 & $4.70\pm 0.01$& $0.444 \pm 0.001$\\ \cline{1-3}
        \end{tabular}
        \caption{Зондовые характеристики при $I_p = 1.5014$ мА и $\text{П}_2$ +}
        \label{tab:my_label}
    \end{table}

        \begin{table}[h]
    \centering
        \begin{tabular}{|c|c|c|}
        \cline{1-3}
        № & I, мкА& V, В\\ \cline{1-3}
        1 & $-35.36\pm 0.01$& $25.011 \pm 0.001$\\ \cline{1-3}
        2 & $-34.11\pm 0.01$& $22.057 \pm 0.001$\\ \cline{1-3}
        3 & $-32.89\pm 0.01$& $19.136 \pm 0.001$\\ \cline{1-3}
        4 & $-31.60\pm 0.01$& $16.017 \pm 0.001$\\ \cline{1-3}
        5 & $-30.28\pm 0.01$& $13.143 \pm 0.001$\\ \cline{1-3}
        6 & $-28.21\pm 0.01$& $10.167 \pm 0.001$\\ \cline{1-3}
        7 & $-25.91\pm 0.01$& $8.092 \pm 0.001$\\ \cline{1-3}
        8 & $-22.55\pm 0.01$& $6.002 \pm 0.001$\\ \cline{1-3}
        9 & $-18.15\pm 0.01$& $4.030 \pm 0.001$\\ \cline{1-3}
        10 & $-12.35\pm 0.01$& $2.015 \pm 0.001$\\ \cline{1-3}
        11 & $-7.62\pm 0.01$& $0.589 \pm 0.001$\\ \cline{1-3}
        \end{tabular}
        \caption{Зондовые характеристики при $I_p = 1.5014$ мА и $\text{П}_2$ -}
        \label{tab:my_label}
    \end{table}

    \begin{tikzpicture}
    \begin{axis}
    [
        xlabel={I, мА}, 
        ylabel={V, В}, 
        grid,
        legend pos=north east,
        height = 0.4\paperheight, 
	width = 0.8\paperwidth,
        title={Вольт-амперная характеристика разряда},
    ]
    
    \addplot[blue, mark=x, only marks] plot [error bars/.cd, y dir = both, y explicit]
    table[x =x, y =y, y error =ey, x error =ex]{V_ot_I1.txt};\
    \addplot[red, mark=x, only marks] plot [error bars/.cd, y dir = both, y explicit]
    table[x =x, y =y, y error =ey, x error =ex]{V_ot_I2.txt};\

    \addlegendentry{Повышение тока разряда}
    \addlegendentry{Понижение тока разряда}
    
    \end{axis}
    \end{tikzpicture}

    \begin{tikzpicture}
    \begin{axis}
    [
        ylabel={I, мА}, 
        xlabel={V, В}, 
        grid,
        legend pos=north east,
        height = 0.4\paperheight, 
	width = 0.8\paperwidth,
        title={Вольт-амперная характеристика разряда},
    ]
    
    \addplot[blue, mark=x, only marks] plot [error bars/.cd, y dir = both, y explicit]
    table[x =x, y =y, y error =ey, x error =ex]{I_ot_V1.txt};\
    \addplot[red, mark=x, only marks] plot [error bars/.cd, y dir = both, y explicit]
    table[x =x, y =y, y error =ey, x error =ex]{I_ot_V2.txt};\

    \addlegendentry{Повышение тока разряда}
    \addlegendentry{Понижение тока разряда}
    
    \end{axis}
    \end{tikzpicture}

    Участок соответствует участку ГД на рисунке 6 приложения к разделу V - поднормальный тлеющий разряд

    \begin{tikzpicture}
    \begin{axis}
    [
        ylabel={I, мкА}, 
        xlabel={V, В}, 
        grid,
        legend pos=north west,
        height = 0.4\paperheight, 
	width = 0.8\paperwidth,
        title={Зондовые характеристики при $I_p = 5.0262$ мА},
    ]
    
    \addplot[blue, mark=x, only marks] plot [error bars/.cd, y dir = both, y explicit]
    table[x =x, y =y, y error =ey, x error =ex]{zond11.txt};\
    \addplot +[mark=none, black] coordinates {(-27, 0) (27, 0)};
    \addplot +[mark=none, black] coordinates {(0, -80) (0, 80)};
    \addplot[black, domain=0:26]{0.9544*x+38.3702};
    

    \addlegendentry{$I_p = 5.0262$ мА}
    
    \end{axis}
    \end{tikzpicture}

        \begin{tikzpicture}
    \begin{axis}
    [
        ylabel={I, мкА}, 
        xlabel={V, В}, 
        grid,
        legend pos=north west,
        height = 0.4\paperheight, 
	width = 0.8\paperwidth,
        title={Зондовые характеристики при $I_p = 3.0094$ мА},
    ]
    
    \addplot[red, mark=x, only marks] plot [error bars/.cd, y dir = both, y explicit]
    table[x =x, y =y, y error =ey, x error =ex]{zond21.txt};\
    \addplot +[mark=none, black] coordinates {(-27, 0) (27, 0)};
    \addplot +[mark=none, black] coordinates {(0, -65) (0, 65)};
    \addplot[black, domain=0:26]{0.517*x+32.1754};

    \addlegendentry{$I_p = 3.0094$ мА}
    
    \end{axis}
    \end{tikzpicture}

    \begin{tikzpicture}
    \begin{axis}
    [
        ylabel={I, мкА}, 
        xlabel={V, В}, 
        grid,
        legend pos=north west,
        height = 0.4\paperheight, 
	width = 0.8\paperwidth,
        title={Зондовые характеристики при $I_p = 1.5014$ мА},
    ]
    
    \addplot[green, mark=x, only marks] plot [error bars/.cd, y dir = both, y explicit]
    table[x =x, y =y, y error =ey, x error =ex]{zond31.txt};\
    \addplot +[mark=none, black] coordinates {(-27, 0) (27, 0)};
    \addplot +[mark=none, black] coordinates {(0, -40) (0, 40)};
    \addplot[black, domain=0:26]{0.2422*x+18.0096};


    \addlegendentry{$I_p = 1.5014$ мА}
    
    \end{axis}
    \end{tikzpicture}

    


    \begin{tikzpicture}
    \begin{axis}
    [
        ylabel={I, мкА}, 
        xlabel={V, В}, 
        grid,
        legend pos=north west,
        height = 0.4\paperheight, 
	width = 0.8\paperwidth,
        title={Зондовые характеристики},
    ]
    
    \addplot[blue, mark=x, only marks] plot [error bars/.cd, y dir = both, y explicit]
    table[x =x, y =y, y error =ey, x error =ex]{zond11.txt};\
    \addplot[red, mark=x, only marks] plot [error bars/.cd, y dir = both, y explicit]
    table[x =x, y =y, y error =ey, x error =ex]{zond21.txt};\
    \addplot[green, mark=x, only marks] plot [error bars/.cd, y dir = both, y explicit]
    table[x =x, y =y, y error =ey, x error =ex]{zond31.txt};\

    \addlegendentry{$I_p = 5.0262$ мА}
    \addlegendentry{$I_p = 3.0094$ мА}
    \addlegendentry{$I_p = 1.5014$ мА}
    \addplot +[mark=none, black] coordinates {(-27, 0) (27, 0)};
    \addplot +[mark=none, black] coordinates {(0, -80) (0, 80)};
    \addplot[black, domain=0:26]{0.2422*x+18.0096};
    \addplot[black, domain=0:26]{0.517*x+32.1754};
    \addplot[black, domain=0:26]{0.9544*x+38.3702};

    
    \end{axis}
    \end{tikzpicture}

    \begin{table}[h]
    \centering
        \begin{tabular}{|c|c|c|}
        \cline{1-3}
        $I_{p}$,  мА & Ветвь& $R_{\text{дифф}}$, МОм \\ \cline{1-3}
        5.0262 & Правая & $1.440\pm0.015$\\ \cline{1-3}
        5.0262 & Левая & $1.426\pm0.023$\\ \cline{1-3}
        3.0094 & Правая & $2.095\pm0.045$\\ \cline{1-3}
        3.0094 & Левая & $1.772\pm0.022$\\ \cline{1-3}
        1.5014 & Правая & $4.835\pm0.159$\\ \cline{1-3}
        1.5014 & Левая & $2.394\pm0.041$\\ \cline{1-3}
        
        \end{tabular}
        \caption{Дифференциальные сопротивления ветвей}
        \label{tab:my_label}
    \end{table}

    \begin{table}[h]
    \centering
        \begin{tabular}{|c|c|}
        \cline{1-2}
        $I_{p}$,  мА & $I_{i\text{н}}$, мкА\\ \cline{1-2}
        5.0262 &  $38.3702\pm1.9865$\\ \cline{1-2}
        3.0094 &  $32.1754\pm0.3381$\\ \cline{1-2}
        1.5014 &  $18.0096\pm0.1938$\\ \cline{1-2}
        
        \end{tabular}
        \caption{Токи насыщения}
        \label{tab:my_label}
    \end{table}

        \begin{table}[h]
    \centering
        \begin{tabular}{|c|c|}
        \cline{1-2}
        $I_{p}$,  мА & $\frac{dI}{dU}$, $10^{-6} \text{Ом}^{-1}$\\ \cline{1-2}
        5.0262 &  $1.535\pm0.004$\\ \cline{1-2}
        3.0094 &  $4.102\pm0.011$\\ \cline{1-2}
        1.5014 &  $3.175 \pm0.010$\\ \cline{1-2}
        
        \end{tabular}
        \caption{$\frac{dI}{dU}$ при U = 0}
        \label{tab:my_label}
    \end{table}
            
            \begin{table}[h]
    \centering
        \begin{tabular}{|c|c|}
        \cline{1-2}
        $I_{p}$,  мА & $T_e$, K\\ \cline{1-2}
        5.0262 &  $145090.5\pm7889.7$\\ \cline{1-2}
        3.0094 &  $45528.3\pm600.5$\\ \cline{1-2}
        1.5014 &  $32924.1\pm457.9$\\ \cline{1-2}
        
        \end{tabular}
        \caption{Температуры электронов при соответствующих $I_p$}
        \label{tab:my_label}
    \end{table}

    \begin{table}[h]
    \centering
        \begin{tabular}{|c|c|}
        \cline{1-2}
        $I_{p}$,  мА & $n_e, \text{м}^{-3}$\\ \cline{1-2}
        5.0262 &  $1.75\cdot 10^16$\\ \cline{1-2}
        3.0094 &  $2.62\cdot 10^16$\\ \cline{1-2}
        1.5014 &  $1.73\cdot 10^16$\\ \cline{1-2}
        
        \end{tabular}
        \caption{Концентрации $n_e$ при соответствующих $I_p$}
        \label{tab:my_label}
    \end{table}

    \begin{table}[h]
    \centering
        \begin{tabular}{|c|c|}
        \cline{1-2}
        $I_{p}$,  мА & $\omega_e, 10^{16} \frac{\text{рад}}{\text{сек}}$\\ \cline{1-2}
        5.0262 &  7.41 \\ \cline{1-2}
        3.0094 &  9.06\\ \cline{1-2}
        1.5014 &  7.37\\ \cline{1-2}
        
        \end{tabular}
        \caption{Частоты колебаний $\omega_e$ при соответствующих $I_p$}
        \label{tab:my_label}
    \end{table}

        \begin{table}[h]
    \centering
        \begin{tabular}{|c|c|}
        \cline{1-2}
        $I_{p}$,  мА & $r_{D_e}$, см\\ \cline{1-2}
        5.0262 &  18.86 \\ \cline{1-2}
        3.0094 &  8.63\\ \cline{1-2}
        1.5014 &  9.03\\ \cline{1-2}
        
        \end{tabular}
        \caption{Радиус дебаевского слоя $r_{D_e}$ при соответствующих $I_p$}
        \label{tab:my_label}
    \end{table}

            \begin{table}[h]
    \centering
        \begin{tabular}{|c|c|}
        \cline{1-2}
        $I_{p}$,  мА & $r_{D}$, см\\ \cline{1-2}
        5.0262 &  0.86 \\ \cline{1-2}
        3.0094 &  0.70\\ \cline{1-2}
        1.5014 &  0.86\\ \cline{1-2}
        
        \end{tabular}
        \caption{Радиус экранирования  $r_{D}$ при соответствующих $I_p$}
        \label{tab:my_label}
    \end{table}


    \begin{table}[h]
    \centering
        \begin{tabular}{|c|c|}
        \cline{1-2}
        $I_{p}$,  мА & $N_D, 10^{10}$\\ \cline{1-2}
        5.0262 &  4.66\\ \cline{1-2}
        3.0094 &  3.76\\ \cline{1-2}
        1.5014 & 4.61\\ \cline{1-2}
        
        \end{tabular}
        \caption{Среднее число ионов в дебаевском слое при соответствующих $I_p$}
        \label{tab:my_label}
    \end{table}

    


\end{document}